\chapter{Introduction}

\section{Motivation}
Specialized areas, such as medical area, convey and use technical words, or terms, which are typically related to knowledge developed within these areas. In the medical area, this specific knowledge often corresponds to fundamental medical notions related to disorders, procedures, treatments, human anatomy, etc. For instance, technical terms like \textit{blepharospasm} (abnormal contraction or twitch of the eyelid), \textit{alexithymia} (inability to identify and describe emotions in the self), \textit{appendicectomy} (surgical removal of the vermiform appendix from intestine), or \textit{lombalgia} (low back pain) are frequently used in the medical area texts.

As in any specialized areas, two main kinds of users exist in the medical area:

\begin{itemize}
    \item medical doctors, both researchers of practitioners, are experts of the domain. They contribute to the creation and development of biomedical knowledge and its exploitation for the healthcare process of patients;
    
    \item  patients and their relatives are consumers of the healthcare process. Usually, they do not have expert knowledge, while it is important that they understand the purpose and issues of their healthcare process. 
\end{itemize}
%%
If the understanding of technical medical terms is easy for the medical staff, patients and their relatives may present some difficulties in the understanding and using of such terms: they show indeed poor \textit{health literacy}. 

Hence, the existing literature provides several studies dedicated to the understanding of medical notions and terms by non-expert users, and on their impact on a successful healthcare process \citep{Mcgray-JAMIA2005,Eysenbach-JMIR2007}. Yet, it is not uncommon that patients and their relatives must face very technical health documents and information. Examples of this kind are frequent and usually the non-expert users are at loss in such situations:

\begin{itemize}
    \item  understanding of information on drug intake \citep{VanderStichele-WILEY1999,Patel-IJMI2002}, such as instructions related to the description and specification of steps necessary for the preparation and intake of drugs,
    
    \item  understanding of clinical documents \citep{Zeng-MEDINFO2007}, which contain important information on the healthcare process of patients,
    
    \item  understanding of clinical brochures or informed consents \citep{Williams-JAMA1995}, which are specifically created for patients and which are typically read by patients during their clinical pathway,
    
    \item  more generally, understanding of information provided for patients by different websites \citep{Oregon-2008,Brigo-EB2015} in different languages (English, Spanish, French) and different medical specialties,
    
    \item  for the same reasons, communication between patients and medical staff \citep{Jucks-HC2007,Tran-EC2009} remains complicated.
\end{itemize}
%%
These various observations provide the main motivation to our work. In this work we address the needs of non-specialized users in the medical domain. As we noticed, the main need is related to the understanding of medical and health information. We propose novel machine learning approaches for a stronger distinction of readability of medical words and distinction of words which may present understanding difficulties to non-expert users. The medical data processed are in French. Seven human annotators participated in creation of the reference data.


\section{Thesis structure}
We first present some related work in chapter \ref{ch:related-work} and background knowledge in chapter \ref{ch:background-information} which form the basis of the methods described in this work. We then introduce the data we used in chapter \ref{ch:dataset-description} and the proposed method in chapter \ref{ch:methodology}. Our results and their discussion are presented in Chapter 6. Finally, we conclude with some directions for future work in Chapter 7.
