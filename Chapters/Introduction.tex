\chapter{Introduction}

\section{Motivation}
Specialized areas, such as medical area, convey and use technical words, or terms, which are typically related to knowledge developed within these areas. In the medical area, this specific knowledge often corresponds to fundamental medical notions related to disorders, procedures, treatments, human anatomy, etc. For instance, technical terms like \textit{blepharospasm} (abnormal contraction or twitch of the eyelid), \textit{alexithymia} (inability to identify and describe emotions in the self), \textit{appendicectomy} (surgical removal of the vermiform appendix from intestine), or \textit{lombalgia} (low back pain) are frequently used in the medical area texts.

As in any specialized areas, two main kinds of users exist in the medical area:

\begin{itemize}
    \item medical doctors, both researchers of practitioners, are experts of the domain. They contribute to the creation and development of biomedical knowledge and its exploitation for the healthcare process of patients;
    
    \item patients and their relatives are consumers of the healthcare process. Usually, they do not have expert knowledge, while it is important that they understand the purpose and issues of their healthcare process. 
\end{itemize}
%%
While the understanding of technical medical terms is easy for the medical staff, patients and their relatives may find it difficult to understand and use such terms. This group of users shows poor \textit{health literacy}. 

The existing literature provides several studies dedicated to the understanding of medical notions and terms by non-expert users, and how the level of health literacy of patients impacts on a successful healthcare process \citep{McCray-JAMIA2005, Eysenbach-JMIR2007}. It is not uncommon that patients and their relatives must face very technical health documents and information. Examples of this kind are frequent, and usually, the non-expert users are at a loss in such situations:

\begin{itemize}
    \item  understanding of information on drug intake \citep{VanderStichele-WILEY1999, Patel-IJMI2002}, such as instructions related to the description and specification of steps necessary for the preparation and intake of drugs,
    
    \item  understanding of clinical documents \citep{Zeng-MEDINFO2007}, which contain important information on the healthcare process of patients,
    
    \item  understanding of clinical brochures or informed consents \citep{Williams-JAMA1995}, which are specifically created for patients and which are typically read by patients during their clinical pathway,
    
    \item  more generally, understanding of the information provided for patients by different websites \citep{Oregon-2008, Brigo-EB2015} in different languages (English, Spanish, French) and different medical specialties,
    
    \item for the same reasons, communication between patients and medical staff \citep{Jucks-HC2007, Tran-EC2009} remains complicated.
\end{itemize}
%%
These various observations provide the main motivation for our work. In this work, we address the needs of non-specialized users in the medical domain. As we noticed, the main need is related to the understanding of medical and health information. 

The recent increase of availability of medical data and rapid spreading of big data analytics tools have facilitated broad application of deep learning techniques in healthcare domain \citep{Jiang-BMJ2017}. The popularity of such methods is due to their ability to mine features from massive datasets of any type (either table, text, image or audio) and result with valuable insights. Classical analytics and machine learning approaches although require less data, need manual feature generation, which involves deep domain understanding, and frequently show much weaker performance in tasks of computer vision and natural language processing \citep{Krizhevsky-NIPS2012, Zhang-NIPS2015}.   

\subsection{The proposed method}
Taking into consideration all above, we propose the following:
\begin{itemize}
    \item applying deep learning techniques for better identification of readability and understandability of medical words by non-expert users. In particular, we will solve a words' categorization task and compare the performance of classification model on different feature sets: standard linguistic and non-linguistic features described in chapter \ref{ch:methodology}, ones obtained using different deep learning approaches and combinations of the previous two.
    \item investigating how different feature sets perform with three different cross-validation settings, described in chapter \ref{ch:experiments}. 
\end{itemize}
The medical data used in this work are in French. Seven human annotators participated in the creation of the reference data (labels specifying understandability of words).

\subsection{Goals of the master thesis}
\begin{enumerate}
    \item To provide an overview of previous works on words' understandability detection.
    \item To apply deep learning techniques for generation of words' features used then by a words' categorization on understandable and not by non-specialists.
    \item To compare the quality of words' categorization from the prospective of understandability on different sets of features and explain the causes of differences in performance.
\end{enumerate}

\section{Thesis structure}
We first present some related work in chapter \ref{ch:related-work} and background information in chapter \ref{ch:background-information} which form the basis of methods described in this thesis. We then introduce the data we used throughout this work in chapter \ref{ch:dataset-description} and the proposed methods in chapter \ref{ch:methodology}. Our results and their discussion are presented in chapter \ref{ch:experiments}. Finally, we summarize our contributions and list the directions for future work in chapter \ref{ch:conclusions}.