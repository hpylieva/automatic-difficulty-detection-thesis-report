\chapter{Related Work}
\label{ch:related-work}

Related work is globally related to text simplification task which involves the detection of complex contents in documents and their adaptation for the target population. In this work, we are interested in the first aspect with additional constraints: detection and diagnosis of technical contents in texts of medical domain. In general non-domain specific context, this task is also known in the literature as complex words identification (CWI). From the overview of related works it will be clear that in the NLP (Natural Language Processing) area, work related to the diagnosis of technical content in general and in the medical domain in particular is quite frequent and topical. 


\section{Early research in readability measurement}

Readability is the ease of understanding written text. The study of readability and how it can be measured takes origin from 1880th with analytics of literature and poetry \citep{Sherman-1893}. Then,  \textit{traditional readability measures} were invented. They rely on two main factors: the familiarity of semantic units such as words or
phrases, and the complexity of syntax. Due to the intention of making these measures straightforward for applications, some simplifying assumptions were used. As a result, final formulas mostly rely on the number of letters and/or of syllables a word contains and on linear regression models \citep{Flesch1948,Gunning1973}. While such readability measures are easy to compute, they are based on shallow characteristics of text, ignoring deeper levels of text processing which are important factors in readability, such as cohesion, syntactic ambiguity, rhetorical organization, and propositional density \citep{Collins-2014}. Moreover, traditional measures of readability were demonstrated to be unreliable for Web pages and other types of non-traditional documents during the recent studies \citep{Si-2001}. As a result of such limitations and due to the recent growth of computational and data resources,  researchers in Natural Language Processing (NLP) area started to work on \textit{computational} readability measurements, which relies on the use of machine learning algorithms on reacher linguistic features.


\section{Data sources}

Machine learning-based approaches require suitable data to result with accurate and usable models. In the detection of difficult for understanding contents creation of data sources is a separate field of study. In recent years, several approaches have been proposed:
\begin{itemize}
\item exploitation of expert judgment, who have an idea on needs of population aimed in the study \citep{DeClerc-NLE2014}. The main limitation is that experts may have difficulties to figure out what are the real needs of the population;
\item exploitation of textbooks created for population according to their readability levels, such as school books \citep{Gala-ELEX2013}. The main limitation is that such books are
  usually created by experts using theoretical basis and observations;
\item exploitation of crowdsourcing involving large population \citep{DeClerc-NLE2014}.  The main limitation is that the population involved is uncontrolled and unknown;
\item exploitation of eye-tracking methods for a more fine-grained
  analysis of reading difficulties \citep{Yaneva-CCA2015,Grabar-ICHI2018}.  The main limitation is that only short text spans can be used;
\item manual annotation by human annotators \citep{Grabar-LREC2016t}. In this case, the annotators represent the population; they are part of the controlled population, they can perform more complicated tasks than in case of crowdsourcing, although they are usually less many than in crowdsourcing experiments. In this work, the data source was constructed using this method. It also was exploited in CWI challenges mentioned in the next section (SemEval-2016 and CWI 2018 Shared Task).
\end{itemize}

Related to this issue is the question on the generalizability of data and models generated from these data.  For instance, it has been observed that data from experts are difficult to generalize over the population \citep{DeClerc-NLE2014}.


\section{Automated Readability Assessment}

\subsection{General language}
For general language, research actions are often performed as part of the NLP challenges. For the case pf CWI for example, there was a shared task on CWI on SemEval-2016 NLP challenge\footnote{\url{http://alt.qcri.org/semeval2016/task11/}}. The goal was to provide a framework for the evaluation of CWI methods, which involved:
\begin{enumerate}
    \item understanding the distinctive characteristics of words which are difficult for non-native speakers;
    \item finding out how well the vocabulary limitations of an individual can be predicted from the knowledge of vocabulary limitations of the group they are part of;
    \item introducing a gold-standard dataset for text simplification and tasks related to topic modeling and semantics.
\end{enumerate}

The participants applied rule-based and/or
machine learning systems, including neural networks for building solutions.
Combinations of various features, designed
to detect the complexity of words, have been used. The most popular among them were: 

\begin{itemize}
    \item simple features: word length, number of syllables, named-entity type, part-of-speech, the position of a word in sentence \citep{Bingel-SemEval2016};
    
    \item number of synsets, senses, synonyms, hyponyms, relations, distinct POSs in WordNet \citep{Ronzano-SemEval2016};
    
    \item corpus-based frequency in large corpora: Wikipedia, Simple Wikipedia \citep{Kauchak-2013}, SubIMDB \citep{Paetzold-SemEval2016solution}, British National Corpus \citep{Ronzano-SemEval2016}, Gigaword corpus and the International Conference on Web and Social Media (ICWSM) blog corpus \citep{Brooke-SemEval2016}. Mostly the frequency was calculated for word-level, but some participants utilized the frequency of char-level n-grams also \citep{Bingel-SemEval2016}.
\end{itemize}

The results of this shared task are described in detail in \cite{Paetzold-SemEval2016overview}. The analysis of 42 submitted systems by 21 teams highlighted that the most effectively CWI task is solved by decision trees \citep{Malmasi-SemEval2016} and ensemble methods \citep{Paetzold-SemEval2016solution, Ronzano-SemEval2016}. Moreover, according to the results, word frequencies remained the most reliable predicting feature of word complexity. The best systems reached up to 77.4 G-score, which measures the harmonic mean between Accuracy and Recall, and 35.3 F-score. 

In this challenge, attempts to apply neural networks showed poor results. Whereas after post-task experiments authors gained competitive results changing the framework of NN implementation, revising architecture and the feature set \citep{Bingel-SemEval2016}. Among features, 300-dimensional GloVe\footnote{\url{https://nlp.stanford.edu/projects/glove/}} word embeddings were found to be the main contributor to NN's performance improvement (from 50.6 to 75.6 G-score). 

Our task is slightly different from the one described in SemEval-2016 Shared task \citep{Paetzold-SemEval2016overview} where given a sentence and a target word within it, the goal is to predict whether or not a non-native English speaker would be able to understand the meaning of the target word. In our formulation we do not have the context near target medical words, so we cannot use it during the training. In other words, the task in SemEval-2016 is CWI in its ordinary meaning, whereas in our case the task comes down to words' classification. The usefulness of standard word embeddings for our task is also not clear, therefore. Moreover, in SemEval-2016 and our task user annotations are made in different languages: English and French correspondingly, -  and have different goals.

After the success of SemEval-2016, the second CWI Shared task\footnote{\url{https://sites.google.com/view/cwisharedtask2018/}} was organized at Building Educational Applications workshop 2018\footnote{\url{http://www.cs.rochester.edu/~tetreaul/naacl-bea13.html}}. This time the data was provided on four languages: English, German, Spanish and French. Whereas, for French, only the test set was available and no French training data. English corpora were extended and involved three genres: news, Wikinews and Wikipedia data. For comparison, on SemEval-2016 the corpora were formed from only Simple Wikipedia data. In 2018 the aim of the CWI Shared
task was to identify challenging for non-native speakers words based on the annotations collected from both native and non-native speakers. 
The analysis \citep{Yimam-BEA2018} of 12 submitted systems and 11 system description papers from 30 teams shows that traditional feature engineering-based approaches (mostly involving words' length and frequency features) still perform better than neural network and word embedding-based approaches. This time much more participants used deep learning approach in their solutions, which resulted in significant improvement of performance in CWI task on monolingual English track: the top rank systems reached from 0.811 to 87.4 F-score for different English datasets. At the same time cross-lingual German, Spanish and French tracks resulted in slightly lower F-score: 74.5, 76.9 and 75.9 correspondingly. Nevertheless, cross-lingual results were considered highly promising. This point was the most important finding of this shared task. 

Among the deep learning solutions used for resolving the CWI 2018 Shared Task  there were:
\begin{itemize}
    \item application of Convolutional Neural Network (CNN) for the first time for CWI task \citep{Aroyehun-BEA2018}. The solution is based on 2D convolution and word-embedding representation of the target text fragment and its context. The CNN-based system did not show significant improvement in performance compared to an alternative system based in feature engineering and Tree Ensembles developed by the same team. 
    
    \item a DNN which was feed with both word-level and character-level embeddings \citep{DeHertog-ACL2018}. The word-level representations were trained by team on their own on COW-corpora\footnote{\url{https://corporafromtheweb.org/}} with gensim\footnote{\url{https://radimrehurek.com/gensim/}} implementation of word2vec model (described in the next chapter \ref{sec:word2vec}). The character-level embeddings were trained by the DNN itself when learning to classify words into complex and non-complex.
\end{itemize}

In contrast to the last solution, in this work, we test the performance of FastText (described in the next chapter \ref{sec:fasttext}), which is a word2vec's modification and captures not only distributional properties of words but also morphological ones, as this model is trained on subword instead of word level. And again, in  CWI 2018 Shared Task words were provided in context, which is different to our setting.


\subsection{Medical area}
Not so much effort has been devoted to the exploitation of NLP potential in the measurement of readability of medical texts.
In the biomedical domain, the readability assessment currently is approached as a classification task as well as in general language. The difference is that here a much smaller variety of features has been tested. The following feature types are mostly used for processing of biomedical documents:
\begin{itemize}
    \item a combination of classical readability formulas
with medical terminologies \citep{Kokkinakis-2006};
    \item n-grams of characters \citep{Poprat-MIE2006}; 
    \item stylistic \citep{Grabar-AMIA2007} or discursive \citep{Goeuriot-LREC2008} features which characterize the discourse of documents;
    \item lexicon features, for example, lexical density - the number of unique number of words within a given unit (e.g. sentence, document) \citep{Miller-HICSS2007};
    \item morphological features \citep{Chmielik-TAL2011};
    \item combinations of different features from the listed above
    \citep{Zeng-MEDINFO2007}.
\end{itemize}

Among the recent experiments dedicated to readability study in the medical domain are, for example, the following: 
\begin{itemize}
    \item manual rating of medical words \citep{Zheng-AMIA2002},
    \item automatic rating of medical words on the basis of their presence in different vocabularies \citep{Borst-MIE2008},
    \item exploitation of machine learning approach
    with various features \citep{Grabar-PITR2014}.
\end{itemize}
The last experiment achieved up to 85.0 F-score on individual annotations.

\bigskip
Due to the recent significant advance in readability study in general language and relatively slow progress with the task in the medical area, there is a great potential of experimenting with the application of various machine learning-based approaches on medical texts. This fact motivated us for for this work.
