\chapter{Related work}
\label{ch:related-work}
Related work is globally related to the detection of technical contents in documents and to their adaptation. Here, we are interested by the first aspect: detection and diagnosis of technical medical contents. 

In the NLP (Natural Language Processing) area, work related to the diagnosis of technical medical documents is quite frequent. Traditionally, researchers exploit the readability measures. Among these measures, it is possible to distinguish classical readability measures and computational readability measures \citep{Francois-TAL2013}. Classical measures usually rely on number of letters and/or of syllables a word contains and on linear regression models \citep{Flesch1948,Gunning1973}, while computational readability
measures may involve vector models and a great variability of
features, among which the following have been used for processing the
biomedical documents: combination of classical readability formulas
with medical terminologies \citep{Kokkinakis-2006}; n-grams of
characters \citep{Poprat-MIE2006}, stylistic \citep{Grabar-AMIA2007} or
discursive \citep{Goeuriot-LREC2008} features, lexicon
\citep{Miller-HICSS2007}, morphological features
\citep{Chmielik-TAL2011}, combinations of different features
\citep{Zeng-MEDINFO2007}.

At a more fine-grained level, the readability of words has been
addressed much less frequently. In the general language, some research
actions are often performed as part of the NLP challenges, such as the
SemEval NLP
challenge\footnote{\url{http://www.cs.york.ac.uk/semeval-2012}} held
in 2012. This challenge proposed the following task: for a short text
and a target word, several possible substitutions satisfying the
context have also been proposed. The objective was to rate and to
order the substitutions according to their degree of simplicity
\citep{specia-SEMEVAL2012}. The participants applied rule-based and/or
machine learning systems. Combinations of various features, designed
to detect the simplicity of words, have been used, such as: lexicon
from spoken corpus and from Wikipedia, Google n-grams, WordNet
\citep{sinha-SEMEVAL2012}; word length, number of syllables, latent
semantic analysis, mutual information and word
frequency\citep{jauhar-SEMEVAL2012}; Wikipedia frequency, word length,
n-grams of characters and of words, random indexing and syntactic
complexity of documents \citep{johannsen-SEMEVAL2012}; n-grams and
frequency from Wikipedia, Google n-grams \citep{ligozat-SEMEVAL2012};
WordNet and word frequency \citep{amoia-SEMEVAL2012}. The best systems
reached up to 0.60 Top-rank and 0.575 Recall.
%%
Another work has been done on scholar texts in French written for
children with the purpose to differentiate between the texts from
various scholar levels and to test various features suitable for that
\citep{Gala-ELEX2013}. This system reached up to 0.62 classification
accuracy.

In the medical area, we can mention three experiments: manual rating
of medical words \citep{Zheng-AMIA2002}, automatic rating of medical
words on the basis of their presence in different vocabularies
\citep{Borst-MIE2008}, and exploitation of machine learning approach
with various features \citep{Grabar-PITR2014}. This last experiment
achieved up to 0.85 F-measure on individual annotations.

Another issue is to know what are the most suitable data for the
analysis of text readability. These data have indeed crutial impact on
models created and on their usability.  Several approaches have been
proposed:
\begin{itemize}
\item exploitation of expert judgment, who have an idea on needs of
  population aimed in the study \citep{DeClerc-NLE2014}. The main
  limitation is that experts may have difficulties to figure out what
  are the real needs of population;
\item exploitation of text books created for population according to
  their readability levels, such as school books
  \citep{Gala-ELEX2013}. The main limitation is that such books are
  usually created by experts using theoretical basis and observations;
\item exploitation of crowdsourcing involving large population
  \citep{DeClerc-NLE2014}.  The main limitation is that the
  population involved is uncontrolled and unknown;
\item exploitation of eye-tracking methods for a more fine-grained
  analysis of reading difficulties
  \citep{Yaneva-CCA2015,Grabar-ICHI2018}.  The main limitation is that
  only short text spans can be used;
\item manual annotation by human annotators
  \citep{Grabar-LREC2016t}. In this case, the annotators represent the
  population, they are part of the controlled population, they can
  perform more complicated tasks than in case of crowdsourcing,
  although they are usually less many than in crowdsourcing
  experiments.
\end{itemize}
%%
Related to this issue is the question on generalizability of data and
of models generated from these data.  For instance, it has been
observed that data from experts are difficult to generalize over the
population \citep{DeClerc-NLE2014}.

We propose novel machine learning approaches for a stronger distinction of readability of medical words and distinction of words which may present understanding difficulties to non-experts users. The medical data processed are in French. Seven human annotators participated in creation of the reference data.