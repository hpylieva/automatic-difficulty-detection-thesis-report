\chapter{Methodology}
\label{ch:methodology}

The purpose of this work is to categorize medical words, or terms, according to whether they can be understood or not by non-specialized people, using features obtained with deep learning methods. The manual annotations of there words provide the reference data. The categorization pipeline is the following: after features for each word are computed, they are used for training the classifiers, and the results are evaluated using the
cross-validation.
Thereby, the proposed method includes three phases: 
\begin{enumerate}
    \item calculation of NLP features associated with the annotated words;
    \item building a machine learning model for words classification;
    \item evaluation of classification quality using cross-validation.
\end{enumerate}

In this research we want to provide answers to the following questions:
\begin{enumerate}
    \item Which feature set distinguishes better between understandable and non-understandable medical words?
    \item Why one feature set categorizes better than another?
    \item Do classifiers built on the considered feature sets generalize well? 
\end{enumerate}


\section{Feature sets}
\subsection{Standard NLP features}

We will refer to the previously used NLP features described in \citep{Grabar-PITR2014} as \textit{"standard features"} (opposed to two kinds of \textit{"embeddings"} described in the next subsection). They include 24 linguistic and extra-linguistic features related to general and specialized languages. The features are computed automatically and can be grouped into ten classes: 

\begin{itemize}
\item {\it Syntactic categories.}  Syntactic categories and lemmas are
  computed by TreeTagger \citep{Schmid-1994} and then checked by Flemm
  \citep{Namer-TAL2000}.  The syntactic categories are assigned to
  words within the context of their terms.  If a given word receives
  more than one category, the most frequent one is kept as feature.
  Among the main categories we find for instance nouns, adjectives,
  proper names, verbs and abbreviations.
\item {\it Presence of words in reference lexica.} We exploit two
  reference lexica of the French language:
  TLFi\footnote{\url{http://www.atilf.fr/}} and {\it lexique.org}\footnote{\url{http://www.lexique.org/}}. TLFi is
  a dictionary of the French language covering XIX and XX
  centuries. It contains almost 100,000 entries. {\it lexique.org} is
  a lexicon created for psycholinguistic experiments. It contains over
  135,000 entries, among which inflectional forms of verbs, adjectives
  and nouns. It contains almost 35,000 lemmas.
\item {\it Frequency of words through a non specialized search
    engine.} For each word, we query the Google search engine in order
  to know its frequency attested on the web.
\item {\it Frequency of words in the medical terminology.} We also
  compute the frequency of words in the medical terminology Snomed
  International.
\item {\it Number and types of semantic categories associated to
    words.} We exploit the information on the semantic categories of
  Snomed International.
\item {\it Length of words in number of their characters and
    syllables.} For each word, we compute the number of its characters
  and syllables.
\item {\it Number of bases and affixes.} Each lemma is analyzed by the
  morphological analyzer D\'erif \citep{Namer-AMIA2004}, adapted to the
  treatment of medical words. It performs the decomposition of lemmas
  into bases and affixes known in its database and it provides also
  semantic explanation of the analyzed lexemes. We exploit the
  morphological decomposition information (number of affixes and
  bases).
\item {\it Initial and final substrings of the words.}  We compute the
  initial and final substrings of different length, from three to five
  characters.
\item {\it Number and percentage of consonants, vowels and other
    characters.} We compute the number and the percentage of
  consonants, vowels and other characters (i.e., hyphen, apostrophe,
  comas).
\item {\it Classical readability scores.} We apply two classical
  readability measures: Flesch \citep{Flesch1948} and its variant
  Flesch-Kincaid \citep{Kincaid-1975}. Such measures are typically used
  for evaluating the difficulty level of a text. They exploit surface
  characteristics of words (number of characters and/or syllables) and
  normalize these values with specifically designed coefficients.
\end{itemize}
%%

\subsection{FastText word embeddings usage}

FastText word embeddings \cite{Bojanowski-ACL2017} is a good candidate as features for words difficulty detection task because they are able to use words' morphological information and generalize over it. The fact that word embeddings capture context and morphological information leads to the hypothesis that incorporating this information as features will improve classification accuracy for our specific problem. FastText embedding vectors are the sum of character n-gram representations, so that they could be generated even for unknown words. As we found out, being trained on Wikipedia and Common Crawl\footnote{\url{http://commoncrawl.org/}} texts the portion of known words from our dataset for current FastText embeddings is quite big. According to our analysis, 44.26\% (13,118 out of 29,641) medical words in the dataset and 56.00\% (16,598 out of 29,641) lowercased medical words in the dataset were used for training of the currently published FastText\footnote{\url{https://fasttext.cc}} model for French.

\subsection{French RNN Medical Undestandibility Text Embeddings}
According to the general RNN idea depicted on \ref{fig:rnn}  
\todo{Add formulas for RNN here or in Capter 3}

FrnnMUTE(French RNN Medical Undestandibility Text Embeddings)




